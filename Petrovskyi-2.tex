\documentclass{/home/alx/Documents/LaTex/nulp}

\begin{document}

\Uttl{Лабораторна Робота №2}{Суміщення зображень на
основі використання дескрипторів}{13}{Пелешко Д.Д.}

\paragraph*{\textit{Тема}} Попередня обробка зображень.

\paragraph*{\textit{Мета}} Навчитись вирішувати задачу суміщення зображень засобом видобування особливих точок і викорисання їх в процедурах матчінгу.

\justify

\section*{Завдання}
Вибрати з інтернету набори зображень з різною контрастністю і різним флуктуаціями освітленості. Для кожного зображення побудувати варіант спотвореного (видозміненого зображення). Для кожної отриманої пари побудувати дескриптор і проаналізувати можливість суміщення цих зображень і з визначення параметрів геметричних перетворень (кут повороту, зміщень в напрямку х і напрямку y).

\begin{enumerate}
\item SIFT
\item PCA-SIFT
\item GLOH
\item DAISY
\item A-KAZE
\item SURF
\item FAST
\item BRISK
\item LDB
\item BRIEF
\item ORB
\end{enumerate}

Для перевірки збігів необхідно написати власну функцію матчінгу, а результати її роботи перевірити засобами OpenCV. Якщо повної реалізації дескриптора не має в OpenCV, то такий необхідно створити власну функцію побудови цих дискрипторів. У цьому випадку матчінг можна здійснювати стандартними засобами (якщо це можливо).


\newpage

\section*{1. Вибір зображень}

Для матчінгу у лабораторній роботі використовувались 3 пари зображень: оригінал і його модифікований фрагмент.

\begin{center}
\Uimg{8}{img/12.jpg}

\textit{Рис.1 Оригінальне зображення}
\end{center}

\begin{center}
\Uimg{3}{img/11.jpg}

\textit{Рис.2 Фрагмент}
\end{center}

\begin{center}
\Uimg{8}{img/22.jpg}

\textit{Рис.3 Оригінальне зображення}
\end{center}

\begin{center}
\Uimg{3}{img/21.jpg}

\textit{Рис.4 Фрагмент}
\end{center}

\begin{center}
\Uimg{8}{img/32.jpg}

\textit{Рис.5 Оригінальне зображення}
\end{center}

\begin{center}
\Uimg{3}{img/31.jpg}

\textit{Рис.6 Фрагмент}
\end{center}

\newpage

\section*{2. PCA-SIFT}

Наступним кроком була реалізація методу метчінгу PCA-SIFT, який є модифікацією SIFT. Як і в SIFT у PCA-SIFT спочатку виявляються масштабно-просторові екстремуми за допомогою побудови пірамід гаусіанів і різниць гаусіанів, локалізація ключових точок і визначення орієнтації. Проте при побудові дескрипторів використовується метод головних компонент (PCA), покликаний видобути найважливіші ознаки, тим самим значно скоротивши результуючу розмірність даних.

Пілся цього може бути проведене зіставлення зображень по ключовим точкам з використанням будь-якого методу. У цій лабораторній був використаний метод FLANN.

\section*{Робота программи}

При запуску програми потрібно передати у аргументах шляхи до фрагменту і цілої картинки. Після цього через деякий час буде виведений результат метчінгу

\begin{center}
\Uimg{10}{results/1.png}

\textit{Рис.7 Результат метчінгу для пари №1}
\end{center}

\begin{center}
\Uimg{10}{results/2.png}

\textit{Рис.7 Результат метчінгу для пари №2}
\end{center}

\begin{center}
\Uimg{10}{results/3.png}

\textit{Рис.7 Результат метчінгу для пари №3}
\end{center}

\newpage

\section*{Аналіз}

Метод PCA-SIFT є модифікацією методу SIFT, у якій використовується метод головних компонент для скорочення обсягу даних і видобування найважливіших особливостей зображень при побудові дескрипторів. Проте, хоча він і має більшу від SIFT швидкість, він повільніший за SURF, та гірше ніж SIFT працює з заблюреними і масштабованими зображеннями.

\section*{Висновки}
У результаті виконання лабораторної роботи я навчився вирішувати задачу суміщення зображень засобом видобування особливих точок і використання їх в процедурах матчінгу.

\end{document}

