\documentclass{/home/alx/Documents/LaTex/nulp}

\begin{document}

\Uttl{Лабораторна Робота №3}{Обробка зображень методами штучного інтелекту}{13}{Пелешко Д.Д.}

\paragraph*{\textit{Тема}} Класифікація зображень.
Застосування нейромереж для пошуку подібних зображень.

\paragraph*{\textit{Мета}} Набути практичних навиків у розв’язанні задачі пошуку подібних зображень на прикладі організації CNN класифікації.

\justify

\section*{Завдання}

Побудувати CNN на основі \textbf{AlexNet} для класифікації зображень на основідатасету fashion-mnist. Зробити налаштування моделі для досягнення необхідної точності. На базі Siamese networks побудувати систему для пошуку подібних зображень в датасеті fashion-mnist. Візуалізувати отримані результати t-SNE.


\newpage

\section*{1. AlexNet}

Нейронна мережа AlexNet складається з 8 шарів: перших 5 - згорткові (використовуються для виділення фіч), останніх 3 - лінійні (використовуються для знаходження закономіронстей і визначення результуючого класу).

\begin{verbatim}
AlexNet = Sequential()

AlexNet.add(Conv2D(filters=96, input_shape=x_train.shape[1:],
    kernel_size=(11,11), strides=(4,4), padding='same'))
AlexNet.add(BatchNormalization())
AlexNet.add(Activation('relu'))
AlexNet.add(MaxPooling2D(pool_size=(2,2), strides=(2,2),
    padding='same'))

AlexNet.add(Conv2D(filters=256, kernel_size=(5, 5), strides=(1,1),
    padding='same'))
AlexNet.add(BatchNormalization())
AlexNet.add(Activation('relu'))
AlexNet.add(MaxPooling2D(pool_size=(2,2), strides=(2,2),
    padding='same'))

AlexNet.add(Conv2D(filters=384, kernel_size=(3,3), strides=(1,1),
    padding='same'))
AlexNet.add(BatchNormalization())
AlexNet.add(Activation('relu'))

AlexNet.add(Conv2D(filters=384, kernel_size=(3,3), strides=(1,1),
    padding='same'))
AlexNet.add(BatchNormalization())
AlexNet.add(Activation('relu'))

AlexNet.add(Conv2D(filters=256, kernel_size=(3,3), strides=(1,1),
    padding='same'))
AlexNet.add(BatchNormalization())
AlexNet.add(Activation('relu'))
AlexNet.add(MaxPooling2D(pool_size=(2,2), strides=(2,2),
    padding='same'))

AlexNet.add(Flatten())
AlexNet.add(Dense(4096, input_shape=(32,32,3,)))
AlexNet.add(BatchNormalization())
AlexNet.add(Activation('relu'))
AlexNet.add(Dropout(0.4))

AlexNet.add(Dense(4096))
AlexNet.add(BatchNormalization())
AlexNet.add(Activation('relu'))
AlexNet.add(Dropout(0.4))

AlexNet.add(Dense(1000))
AlexNet.add(BatchNormalization())
AlexNet.add(Activation('relu'))
AlexNet.add(Dropout(0.4))

AlexNet.add(Dense(10))
AlexNet.add(BatchNormalization())
AlexNet.add(Activation('softmax'))
\end{verbatim}

\section*{2. Siamese Networks}

Сіамські мережі побудовані наступним чином: використовуються 2 AlexNet з окремими входами, спільним виходом і вагами. На виході кожна AlexNet продукує вектори передбачень, які можна порівнювати між собою з метою визначення схожості вхідних зображень.

\begin{verbatim}
img_a_in = Input(shape=x_train.shape[1:], name='ImageA_Input')
img_b_in = Input(shape=x_train.shape[1:], name='ImageB_Input')
img_a_feat = AlexNet(img_a_in)
img_b_feat = AlexNet(img_b_in)
combined_features = concatenate(
    [img_a_feat, img_b_feat], name='merge_features')
combined_features = Dense(16, activation='linear')(combined_features)
combined_features = BatchNormalization()(combined_features)
combined_features = Activation('relu')(combined_features)
combined_features = Dense(4, activation='linear')(combined_features)
combined_features = BatchNormalization()(combined_features)
combined_features = Activation('relu')(combined_features)
combined_features = Dense(1, activation='sigmoid')(combined_features)
similarity_model = Model(inputs=[img_a_in, img_b_in], outputs=[
                         combined_features], name='Similarity_Model')
\end{verbatim}

\section*{3. Експериментальна частина}

Випадково ініціалізовані сіамські мережі не дали ніякого корисного результату.

\begin{center}
\Uimg{6}{1.png}
\end{center}

Проте після 25 епох тренування моделлю була досягнута точність близька до 80\%.

\begin{center}
\Uimg{3}{5.png}
\end{center}

\begin{center}
\Uimg{6}{2.png}
\end{center}

Використання t-SNE продемонструвало, що моделі, що генерували фічі (AlexNet), змогли досить добре виокремити ознаки чобіт, пуловерів, штанів, сумок, сандалей та кросівок (які є візуально відмінними і легшими до розрізнення). Проте, вони з меншою впевненістю розрізняють сукні, плащі, сорочки і футболки (ймовірно через зовнішню схожість). Також деякі плаття плутаються з сумками.

\begin{center}
\Uimg{8}{4.png}
\end{center}

\section*{Висновки}
У результаті виконання лабораторної роботи я набув практичних навиків у розв’язанні задачі пошуку подібних зображень на прикладі організації CNN класифікації.

\end{document}

