\documentclass{/home/alx/Documents/LaTex/nulp}

\begin{document}

\Uttl{Лабораторна Робота №1}{Обробка зображень методами штучного інтелекту}{13}{Пелешко Д.Д.}

\paragraph*{\textit{Тема}} Попередня обробка зображень.

\paragraph*{\textit{Мета}} Вивчити просторову фільтрацію зображень, методи мінімізації шуму, морфології, виділення країв і границь та елементи біблотеки
OpenCV для розвязання цих завдань.

\justify

\section*{Хід роботи}
\begin{enumerate}
\item Вибрати з інтернету два зображення з різною деталізацією об’єктів та два зображення з різним контрастом.
\item Виконати детекцію границь на зображеннях за допомогою операторів Sobel,
Prewitt.
\item Провести порівняльний аналіз.
\end{enumerate}

\newpage

\section*{1. Вибір зображень}

Для того, щоб відмнінності роботи методів детекції границь були більш помітними, було вирішено обрати 1 картинку і власноруч підвищити/знизити контрастність та деталізацію.

\begin{center}
\Uimg{10}{img/original.jpg}

\textit{Рис.1 Оригінальне зображення}
\end{center}

\begin{center}
\Uimg{10}{img/low.png}

\textit{Рис.2 Низька контрастність}
\end{center}

\begin{center}
\Uimg{10}{img/high.png}

\textit{Рис.3 Висока контрастність}
\end{center}

\begin{center}
\Uimg{10}{img/soft.png}

\textit{Рис.4 Низька деталізація}
\end{center}

\begin{center}
\Uimg{10}{img/sharp.png}

\textit{Рис.5 Висока деталізація}
\end{center}

\newpage

\section*{2. Методи детекції}

Наступним кроком була реалізація методів детекціі границь операторами Собеля і Прюітт.

\begin{verbatim}
def Sobel(A):
    # kernel for tracking horizontal changes
    Kx = pl.array([
        [-1, 0, +1],
        [-2, 0, +2],
        [-1, 0, +1],
    ])

    # kernel for tracking vertical changes
    Ky = pl.array([
        [+1, +2, +1],
        [0,  0,  0],
        [-1, -2, -1],
    ])

    # partial gradient magnitudes
    Gx = convolve2d(A, Kx, mode='valid')
    Gy = convolve2d(A, Ky, mode='valid')

    # full gradient magnitude
    G = pl.sqrt(pl.square(Gx) + pl.square(Gy))
    G *= 255.0 / G.max()

    return G
\end{verbatim}

\newpage

\begin{verbatim}
def Prewitt(A):
    # kernel for tracking horizontal changes
    Kx = pl.array([
        [+1, 0, -1],
        [+1, 0, -1],
        [+1, 0, -1],
    ])

    # kernel for tracking vertical changes
    Ky = pl.array([
        [+1, +1, +1],
        [0,  0,  0],
        [-1, -1, -1],
    ])

    # partial gradient magnitudes
    Gx = convolve2d(A, Kx, mode='valid')
    Gy = convolve2d(A, Ky, mode='valid')

    # full gradient magnitude
    G = pl.sqrt(pl.square(Gx) + pl.square(Gy))
    G *= 255.0 / G.max()

    return G
\end{verbatim}

Завантаження картинок і перетворення до вигляду grayscale відбувається з використанням бібліотеки OpenCV.

\begin{verbatim}
high = cv2.imread('img/high.png', 0)
low = cv2.imread('img/low.png', 0)
sharp = cv2.imread('img/sharp.png', 0)
soft = cv2.imread('img/soft.png', 0)
\end{verbatim}

\section*{Робота программи}

Після запуску результати детекції границь двома методами будуть послідовно виведені для кожної з картинок.

\begin{center}
\Uimg{10}{1.png}

\textit{Рис.6 Результати детекції}
\end{center}

\section*{Аналіз}

Оператори Собеля і Прюітт однаково застосовуються для визначення границь зображення шляхом знаходження градієнтів в вертикальному та горизонтальному напрямках з подальшим їх накладанням. Єдина їх відмінність - це ядро згортки. У операторі Прюітт крайні горизонтальні і вертикальні ряди містять одиницю, а в операторі Собеля по центру ненульових рядків стоять двійки, що робить акцент на пікселях, які ближче до центру фільтра.

У ході візуальних порівнянь обох методів було виявлено, що границі, отримані в результаті застосування методу Собеля трохи світліші (за рахунок двійок у ядрі), проте суттєвих відмінностей між результатами детекції границь не було помічено.

\section*{Висновки}
У результаті виконання лабораторної роботи я вивчив просторову фільтрацію зображень, методи мінімізації шуму, морфології, виділення країв і границь та елементи біблотеки OpenCV для розвязання цих завдань.

\end{document}

